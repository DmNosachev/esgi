
\label{authors:start}

\renewcommand{\esgiSecName}{К сведению авторов}

\renewcommand{\esgiSecNameEn}{For the Authors}

\vspace{-1em}
\esgiSection{}{\esgiSecName}

\esgiSectionContEn{}{\esgiSecNameEn}

\begin{center}
    Правила оформления рукописей
(действуют с 1 марта 2025 г.)
\end{center}


%\vspace{2em}


\begin{multicols}{2}

\setstretch{1.0}

    \textbf{ВНИМАНИЕ! Для публикации статьи
    в журнале автор оформляет подписку на
    2 номера журнала}. Онлайн-подписка оформляется по каталогу «Пресса России». Подписной индекс: \textbf{80114}. Ссылка на каталог:
    \url{https://www.pressa-rf.ru/cat/1/edition/t80114}

    Научный журнал «Экономические и социально"=гуманитарные исследования» публикует на русском языке оригинальные и обзорные статьи. Основные рубрики:
\begin{itemize}
    \item  экономика инновационного развития:
    теория и практика;
    \item  философия: мир в человеке и человек
    в мире;
    \item  педагогическая система координат: образование, воспитание, развитие человека.
\end{itemize}    

    В редакцию предоставляются:
\begin{enumerate}
    \item текст статьи (подписанный всеми авторами, допускается электронная подпись
    в формате pdf), включая список авторов, название, аннотацию, рисунки, таблицы, библиографический список;
    \item  анкеты авторов (см. бланк анкеты на
    сайте журнала; адрес сайта: \url{http://esgi-miet.ru});
    \item  рекомендации кафедры; сопроводительное письмо на официальном бланке
    (для аспирантов из сторонних организаций).
\end{enumerate}
    Ориентировочный объем публикаций: для
    статьи — не менее 8—10 страниц текста
    (от \num{20000} до \num{40000} знаков); материалы объемом менее 6 страниц текста (\num{12000} знаков)
    рассматриваются как краткие сообщения.
    Материал для публикации должен быть
    собран в один файл с названием \textbf{ФамилияИО\_Название статьи}.

    \textbf{Внимание!} Все поступающие материалы
    проходят проверку в программе «Антиплагиат». Оригинальность текста: не менее 78\%;
    самоцитирование не более 10\%; цитирование не более 20\%.

    \textbf{Оформление первой страницы статьи}:
    индекс УДК; название статьи; инициалы,
    фамилия автора; название учреждения, где
    выполнена работа; аннотация на русском
    и английском языках; ключевые слова. Далее
    следует текст статьи. (Подробнее см.: шаблон
    оформления статьи на сайте журнала. Адрес
    сайта: \url{http://esgi-miet.ru})
    
    \textbf{Содержание статьи} должно соответствовать тематическому направлению и научному уровню журнала, обладать определенной новизной и представлять интерес для
    широкого круга читателей.

    Авторам настоятельно рекомендуется
    структурировать текст статьи: выделить
    вводную часть, описание материалов и методов исследования (изложение теоретических основ, обзор основных теорий), результаты и их обсуждение, сделать выводы.
    
    {Аннотация} (описательная) предоставляется на русском и английском языках,
    должна включать характеристику исследования с освещением его основных вопросов:
    предмет, основные гипотезы, результаты
    и выводы. Рекомендуется использовать отработанные клише: рассмотрены, изучены,
    представлены, проанализированы, обоснованы, показаны и др. Объем аннотации — 
    до 150 слов.
    
    {Ключевые слова} или словосочетания
    должны отвечать тематике исследования, соответствовать тематике статьи. Приводятся
    на русском и английском языках, отделяются
    друг от друга запятой, в конце точка не ставится.
    
    {Рисунки} дополнительно предоставляются в отдельных файлах; они должны
    быть черно"=белыми или в градациях серого.
    Векторные рисунки предоставляются в любом
    из форматов pdf; eps; ai. Растровые рисунки
    (фотографии) — в любом из форматов jpeg;
tiff; png; psd. Разрешение 300 точек на дюйм,
ширина рисунка ≤ 160 мм.

При выборе единиц измерения следует
руководствоваться утвержденной системой
единиц физических величин (см. ГОСТ
\mbox{8.417-2002}).

\textbf{Географические названия} должны соответствовать атласу последнего года издания.

В тексте \textbf{ссылки на цитируемую литературу} даются в круглых скобках по образцу:
(Автор, год: страницы), например: (Kotler,
2023: 41—58). Список литературы и источников не нумерованный, оформляется в порядке алфавита фамилий авторов и (или) первых слов названий (слово за словом). Источники на иностранных языках располагаются
в конце списка и выстраиваются в соответствии с латинским алфавитом. Рекомендуется
использовать не более 15 (опубликованных)
литературных источников для оригинальной
статьи, не менее 30 источников для обзорной
статьи. При необходимости число ссылок
в конкретной рукописи может быть скорректировано по согласованию с редакцией.

{Библиографическое описание} оформляется по 7-му изданию стандарта MLA
(\url{https://libguides.heidelberg.edu/mla7/home})
при помощи любого удобного библиоменеджера или функционала, встроенного в Microsoft Office (вкладка «Ссылки» — кнопка «Вставить ссылку»). Необходимо указать:
\begin{itemize}
\item для книг: фамилию и инициалы авторов
(всех или 6--7 первых), полное название
книги, место издания, издательство, год,
том или выпуск, ссылку на конкретные
страницы;
\item  для периодических изданий: фамилию
и инициалы авторов всех или 6--7 первых), название статьи, название журнала,
год издания, том, номер, страницы публикации;
\item  для материалов конференций, школ, семинаров: фамилию и инициалы докладчиков, название доклада, время и место
проведения конференции (мероприятия), название конференции (мероприятия), город, издательство, год, страницы
публикации;
\item  для электронных ресурсов: сведения об
авторстве (если есть), название, год,
номер (если есть), URL, дату обращения.
Список авторов и сведений о них должен
содержать:
\item  информацию о каждом авторе для публикации (на русском языке) — фамилия,
имя, отчество (полностью), ученая степень, ученое звание, место работы (полное название организации), занимаемая должность, идентификаторы (SPIN,
ORCID и т.\,п.), если есть;
\item  e-mail для публикации в интернете.
Необходимо также предоставить контактную информацию (не для публикации) — 
телефон, адрес электронной почты.
\end{itemize}

В статье, подготовленной несколькими
авторами, следует указать ответственного
за прохождение статьи, для аспирантов —
научного руководителя. Все сведения
должны соответствовать указанным
в авторской анкете.

\textbf{Требования к оформлению текста}: шрифт
Times New Roman, кегль 14, интервал
полуторный, выравнивание по ширине,
поля 2 см, отступ 1,25. Для форматирования
текста не следует использовать повторяющиеся пробелы и знаки табуляции. Необходимо различать дефис (-), знак «минус» (−)
и тире (—). Нумерация рисунков и таблиц
сквозная. Единственная таблица, единственный рисунок не нумеруются. Мелкие
формулы выполняются в текстовом редакторе, а крупные в редакторе формул.
Использование аббревиатур предполагает
расшифровку.

Решение о публикации или отклонении рукописи принимается редколлегией
по результатам анонимного рецензирования.

Рукописи, не соответствующие указанным требованиям, редакцией не рассматриваются.

\begin{flushleft}
    Статьи направлять по адресу: \bfseries 124498,
    Москва, пл. Шокина, д. 1, МИЭТ,
    редакция журнала «Экономические
    и социально"=гуманитарные исследования».\\
    Е-mail: \href{mailto:esgi.miet@yandex.ru}{esgi.miet@yandex.ru} \par
\end{flushleft}

\end{multicols}

\setmainlinespread

\label{authors:end}